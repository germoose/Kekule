\documentclass[12pt]{article}
\usepackage{pslatex} % Font Times New Roman

\usepackage{amsmath,amssymb,amsfonts,fancyhdr,lastpage}
\usepackage{makeidx, graphicx}

\setlength{\textheight}{9.0in}
\setlength{\topmargin}{-.5in}
\setlength{\textwidth}{6.5in}
\setlength{\oddsidemargin}{-.0in}
\setlength{\parskip}{6pt}


\renewcommand{\headrulewidth}{0pt}
\setlength{\headsep}{.35in}
\addtolength{\headheight}{2.5pt} 

\usepackage{graphicx}

\begin{document}

\title{Representing Kekule Cells in Stable Molecules \\ (USRA Research Proposal)}
\author{Aaron Germuth and Alex Aravind \\\\  
Department of Computer Science \\
University of Northern British Columbia \\
E-mail: (germuth,csalex)@unbc.ca}
\maketitle


\begin{abstract}

This report outlines the research work that we propose to conduct in the next three months, June - Aug. 2013.

\end{abstract}

\section{Introduction}

Carbon is the $6^{th}$ element of the periodic table and backbone of all organic molecules. It is tetravalent, meaning it has 4 valence electrons available to form covalent chemical bonds. In such bonds, atoms share electrons relatively equally between them. Each possible bonding partner of carbon will contribute a single electron to the covalent bond, meaning carbon, in its full valency will have 8 electrons. This is considered a full octet, and the stable configuration of carbon. However, 4 bonding partners is not always needed in order to be stable. If Carbon was bonded to 3 other atoms, but one of them was able to share 2 electrons (a double bond), carbon would still have a full valency of 8 electrons.

According to the Valence Bond Theory, such single and double bonds have different electron structures. Single bonds are named ``sigma ($\sigma$) bonds", and are caused by the head-on overlapping between atomic orbitals. Sigma Bonds are the strongest type of covalent bond. Double bonds are named ``pi ($\pi$) bonds", and result from electron overlap in the nodal plane which passes through both of the bonding atoms. Double bonds are weaker, but still resilient.

Many organic molecules display alternating paths of single and double bonds. In this configuration, every atom is connected with precisely one double bond. This is named conjugation, and can increase molecular stability. Conjugation implies the possibility of different resonance structures, where every double bond shifts in one direction to obtain a different representation of the same molecule. The amount of such resonance structures is a measure of stability. The actual molecule results from the superposition of all resonance contributors.

Each of the nodal lobes involved in pi bond conjugation allow their electrons to delocalize across the molecule. Since electrical current is the flow of electrons, such alternating paths are postulated to be electrically conductive \cite{H13,HK88}. When electricity (specifically a `soliton') is ran through a conjugated system, all single and double bonds are interchanged \cite{HK88}. 

This allows some molecules to display a `switching' property. A channel is defined as a path through a molecule. A channel is open when the path is alternating, and closed otherwise. When some alternating paths are toggled, other paths may be opened or closed. This resembles the behavior of circuitry, and is proposed as a basis for molecular computation. This proposal is quite speculative.

\section{Literature Review}

	Hesselink et al. \cite{H13,HH13} has previously shown that the switching behaviour can be completely described using Kekule Theory. Graph theory is used to model cyclic unsaturated hydrocarbons, where atoms are abstracted as nodes, and edges represent chemical covalent bonds. These graphs neglect all hydrogen atoms, as they do not contribute to conjugation. 

Double bonds are represented using a pairwise disjoint subgraph, where every edge in the subgraph represents a double bond. Pairwise because a double bond reaches two vertices, and disjoint because double bonds are seen at every other position in an alternating path. It is possible to give every carbon atom precisely one double bond if its graph has a `perfect matching'. Each perfect matching corresponds to a single resonance structure. 

Certain vertices are labelled `ports' are are not required to contain a double bond. Therefore, our matchings can contain imperfections at the ports.  Ports are where the molecule connects to the `outside world'. This could allow not only connectivity between molecules, but locations to observe and influence molecular behaviour. Anything can be attached to a port, as it does not affect the interchanging of bonds or the switching behaviour.

Each graph is associated with a Kekule cell, which consists of all the possible stable double bond configurations of that molecule. This is recorded by naming the ports, and labelling each configuration based on which ports contain double bonds. From this, Hesselink has classified all possible Kekule cells with ports $ \leq 5$. He has also classified 210 out of 214 possible Kekule cells with 6 ports.This means nearly any molecule with 6 or less ports has the exact same `switching behavior' as one of the structures found by Hesselink. However, results obtained by Hesselink are graphs which only represent the Kekule cell, not graphs which also resemble molecules. 
	                 

\section{Motivation and Proposed Work}

We derive our motivation from Prof. Hesselink's work. Let us start with a quote. 	
``For application to carbon chemistry, it would be interesting to see whether all Kekule cells ... can be realized in stable molecules" 	- Hesselink, 2013 \cite{H13}.

	This area of research is relatively new. To our knowledge, of the few studies that have been done, none of them have made any attempt to obtain realistic graphs. For our contribution, we would like to first, create our own software framework to confirm previous results. We would then like to attempt to find alternate more realistic structures for each one given in Hesselink’s classification. 

	If we want a molecule with a certain switching behaviour, the search can be split into three sections:
	
\begin{enumerate}
\item Search for cells K with the required switching behavior.
\item Search for graphs G which have cell K.
\item Search for suitably alternating hydrocarbon molecules that have such a graph. 	- Hesselink, 2013 \cite{H13}.
\end{enumerate}

	Our research attempts to bridge the gap between step 2 and 3, now that Hesselink [1] has made strong headway in step 2.  We approach the problem first, by adding multiple restrictions to our resulting graphs.

``In view of the application to conjugation in carbon chemistry, we could restrict attention to graphs where all nodes degrees  $ \leq  4$" 	- Hesselink et al., 2007 \cite{HH13}. 

Hesselink himself has said that graphs could be restricted so nodes have degrees of 4 or less. However, we feel this is not strict enough. Every atom in such conjugated systems is involved in a double bond (or else the structure wouldn’t be stable). Since double bonds are seen only in subgraphs, a node should be at maximum allowed to connect to 3 distinct vertices. 

What about ports? Ports aren't required to have a double bond in order for the structure to be stable. However, in most molecules, at least one configuration involves a double bond at any given port. Additionally, ports are defined as atoms which ``other chemical groups can be attached to observe and influence behavior" \cite{H13}. This means ports must have: (1) The capacity to form a double bond, and (2) One bond reserved to connect to outside chemical groups. Therefore we restrict ports to at maximum, be allowed to connect to 2 distinct vertices. 
Other than the degree of vertices, there are other inconsistencies with some of Hesselink’s  results \cite{H13}. Graphs are permitted to be disjoint. An unconnected graph really represents two molecules, not one, with no feasible means of conjugation between them. Therefore, results must also be limited to connected graphs. Other unstable structures such as  3 membered rings of carbon (which appear in some graphs) are sterically unfavourable and should be avoided.

\section{Challenges}

There are many challenges with such an approach. How do we find a graph for a given cell? The current approach involves searching for a smaller graph whose cell is a subset of the required cell. We then add edges to our graph, slowly approximating our final cell. It is not currently known whether it is possible to make an algorithm which determines whether a given cell is Kekule. Hesselink's \cite{H13} program is capable of enumerating all graphs, and testing each of them to see if their cell matches the input. However, if a cell was not Kekule, Hesselink's program would search forever. Therefore the class of Kekule cells is currently semi-decidable.

Additionally, the time complexity of the current approach is quite large. It takes many hours just to classify all cells of order 6. For this reason, Hesselink has used a finite set of verticies and edges. This also means his algorithm will always terminate, although it will not always give a solution when it does. When searching for cells of order 6, Hesselink was not able to find graphs for 4 of them because of the limited vertex/edge set. However, allowing more components in the graph is computationally prohibitive due to the large growth rate. Considering we intend to search for graphs with much more restrictions than Hesselink, it is likely some graphs will be unfindable.

This suggests the possibility of rather than searching for restricted structures, to simply edit Hesselink's structures. The editing of a graph without changing its Kekule cell has been studied in \cite{H13,HH13,v06}. Hesselink et al. \cite{HH13} shows a way to split a node of high degree into two nodes of lower degree, and even shift edges across verticies. M.H. van der Veen \cite{v06} outlines eight topological operations used algorithmically to create omniconjugated systems (alternating path between all ports). Perhaps its possible to create a similar algorithm which edits graphs towards our restrictions rather than omniconjugation.
	
\section{Work Plan}

First we intend to completely familiarize ourselves with Hesselink's work. This will involve studying his C program at the theoretical level, as well as being comfortable with its execution. We will then begin to create our own software framework in Java. We will confirm Hesselink's results and attempt to add restrictions of our own such as limited degrees and connectedness. 
Based on our results, we will begin to edit our new graphs to remove steric hinderance and other inconsistencies between our graphs and organic molecules. In order for us to state our graphs represent realistic molecules, we will likely need to search for existing organic molecules which resemble our generated structures.


\begin{thebibliography}{abrv}

\bibitem{H13} W. H. Hesselink, Graph Theory for alternating hydrocarbons with attached ports. Indagationes Mathematicae, Elsevier, 24:115141, 2013.
\bibitem{HH13} W.H. Hesselink, J.C. Hummelen, H.T. Jonkman, H.G. Reker, G.R. Renardel de Lavalette, M.H. van der Veen, Kekule Cells for Molecular Computation. Cornell University Library Online, 2013.
\bibitem{v06} M.H. van der Veen. $\pi$-Logic. PhD thesis, University of Groningen, May 2006.
\bibitem{HK88} A.J. Heeger, S. Kivelson, J.R. Schrieffer, and W.-P Su. Solitons in conducting polymers, Rev. Mod. Phys.,60:781, 1988.

\end{thebibliography}

\end{document}
